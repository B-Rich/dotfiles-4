\documentclass[11pt]{ltxdoc}
\usepackage{invoice}
\usepackage{pslatex,hyperref}

\title{\texttt{invoice \InvoiceVersion}\\
	A Package for Writing Invoices}
\author{Oliver Corff}
\date{December 16th, 2003}
\begin{document}
\maketitle

\tableofcontents

\section{Introduction}

The \texttt{invoice} package was conceived in late 2000 when the
author had to dig through a truly aweful pile of expense bills
without having a spreadsheet featuring \LaTeXe-compliant output (or
any spreadsheet, for that purpose) available. After several
miscalculations with a pocket calculator due to forgotten or
double entries the idea came up to have \LaTeXe\ do the calculation
work. As such, the package in its present stage is highly
specialized with regard to the documents it generates. The
\texttt{invoice} package is basically a tailor-made solution for a
consultant who charges fees and claims all sorts of expenses,
sometimes working on different assignments for the same client.

The author expresses his gratitude to
Robert Inder,
Thilo Barth,
Jacco Kok,
Fred Donck,
Jacopo,
Johann Spies,
Ian Wormsbecker,
Vincent Tougait
and
Robin Fairbairns
who contributed ideas, corrections,
bugfixes and caption translations after the first discussions on
\texttt{comp.text.tex} and the initial release of \texttt{invoice}.

Given the current capabilities of \texttt{invoice}, it should well
be possible to extend the capabilities of this package in the future
or to rewrite it in a generalized fashion.

\section{Software Requirements}

The \texttt{invoice} environment runs under \LaTeXe\ and relies on the
\texttt{calc.sty} (providing infix arithmetic) and \texttt{realcalc}
(providing real arithmetic) utilities to do its work which can be found
at CTAN%
	\footnote{The \texttt{realcalc} package is found at
	\texttt{CTAN:macros/generic/realcalc},
	and \texttt{calc} is found at
	\texttt{CTAN:macros/latex/required/tools/}.}.
Compile and read \texttt{00README.tex} for further information if
you are not sure whether these packages are installed at your site.

\section{The \texttt{invoice} Environment}

Within a given document, invoices are built with the
\texttt{invoice} environment\footnote{Users of the KOMA-Script 
	class \texttt{scrlettr.cls} are kindly requested to use
	\texttt{invoiceenv} instead; see also page~\pageref{koma}.}.
Figure~\ref{structure} on page~\pageref{structure} shows the
logical structure of an invoice as well as its basic commands.
In case a full-fledged invoice stationary is needed, it is recommended
to use the \texttt{invoice} environment within existing business 
letters which may have been pre-defined already. The \texttt{invoice}
package itself does not provide tools for including company logos,
recipient's addresses, bank account numbers etc. as these are usually
covered by the various \texttt{letter} classes available for \LaTeXe.

\begin{figure}[h]
\begin{center}
\fbox{
\begin{minipage}{8.5cm}%
\textbf{\huge --- Invoice ---\\[2mm]}
	\texttt{\Large \char92 begin\{invoice\}\{...\}\{...\}\\[2mm]}
	\fbox{
	\begin{minipage}{7.0cm}
		\textbf{\Large --- Project ---\\[1.5mm]}
		\texttt{\large\char92 ProjectTitle\{...\}\\[1.5mm]}
		\fbox{
		\begin{minipage}{5.5cm}
			\textbf{\large --- Fees ---\\}
			\texttt{\char92 Fee\{...\}\{...\}\{...\}}\\
			...\\
			...\\
		\end{minipage}
		}\\[2mm]
		\fbox{
		\begin{minipage}{5.5cm}
			\textbf{\large --- Expenses  (local) ---\\}
			\texttt{\char92 EBC\{...\}\{...\}}\\
			...\\
			...\\
			\textbf{\large --- Expenses (foreign) ---\\}
		\texttt{\char92 EFC\{...\}\{...\}\{...\}\{...\}\{...\}}\\
			...\\
			...\\
		\end{minipage}
		}\\[2mm]
	\end{minipage}
	}\\[2mm]
	\fbox{
	\begin{minipage}{7.0cm}
		\textbf{\Large --- \dots\ More Projects\dots\ ---}
	\end{minipage}
	}\\[2mm]
%	\fbox{
%	\begin{minipage}{7.0cm}
%		\textbf{\Large --- Project ---}
%	\end{minipage}
%	}\\
	\texttt{\Large\char92 end\{invoice\}}
\end{minipage}
}
\end{center}
\caption{The \texttt{invoice} Environment
	and its Logical Structure\label{structure}}
\end{figure}

Invoices contain one or more projects which in return contain the charged
items, either fees (plus tax, if applicable) and/or expenses. An
invoice with one project is announced by saying 

\begin{verbatim}
\begin{invoice}{<Base Currency>}{<VAT>}
	\ProjectTitle{...}%
\end{invoice}
\end{verbatim}

There is no limit for the number of projects in an invoice, as there
is no limit for the number of invoices per document.

The \texttt{invoice} environment requires two arguments: 
\begin{enumerate}
	\item \texttt{<Base Currency>} is the name of the currency
		in which the invoice is charged, e.\,g. DM, Euro,
		US\$, RMB etc.
	\item \texttt{<VAT>} is the percentage rate of VAT which is charged;
		in Germany this is currently (winter 2003) 16\%. If
		no VAT is required, enter a \texttt{0}
		(\textit{zero}) here. It is neither necessary nor
		permissible to use a percent sign here. As some
		countries (e.\,g. France) have fractions of
		percentages (like 16.9\%), you would in this case
		enter \verb|16.9| (without any percent sign).

		Setting the VAT rate to \verb-0- produces the side
		effect that the lines stating the VAT subtotals and
		totals disappear.

		If it is, however, desired to show the VAT results
		even if they amount to zero, enter \verb-0.0- instead.
\end{enumerate}


\subsection{Projects}

An invoice contains items which are usually, in the case of e.\,g.
consultancy fees and related expenses like hotel bills and air
fares, attributed to a given case or \textit{project}, or
cost center, or ``Kostenstelle'' (in German).

A project contains any of three different types of charged items:
\begin{enumerate}
	\item \textbf{Fees}. A tax can be added, if applicable. Fees 
		are always charged in the
			base currency
		of the invoice.
	\item \textbf{Local Expenses}. Local expenses are charged in 
		units of the 
			base currency
		of the invoice.
	\item \textbf{Foreign Expenses}. Foreign expenses are charged in 
		units of any given foreign currency. Either the
			base currency
		equivalent is known (as taken from a credit card
		billing statement, for example), or, if not, an
		exchange rate between foreign currency and base
		currency has to be stated.
\end{enumerate}

\textbf{Nota bene:} The order of fees and expenses is fixed. Either fees or
expenses can be omitted, but expenses must be charged \textit{after}
fees.


\section{The First Example: How to Charge Fees}


A consultant charges fees per day, hour or any other unit. Usually
this unit is agreed upon in a contract and there is no further need
to refer to this unit but by its count. This is done by the
\verb-\Fee{}{}{}- command:

\begin{verbatim}
	\Fee{<Contents>}{<Rate/Unit>}{<Count>}
\end{verbatim}

Let's assume an interim manager
charges DM 1818.00 a day for 12 working days while
negotiating a major project, nicknamed \textit{Project Phenix}.
He also charges DM 2750.00 a day for analysing and negotiating
the restructuring of the sales division, a work he spent 9 days with.

\subsection{Invoices with VAT}

We further assume that the consultant is required to charge VAT.
All information above would be entered into the invoice as follows:

\begin{verbatim}
\begin{invoice}{DM}{16}
   \ProjectTitle{Project Phenix}%
       %    Contents                    Rate/Unit   Count
       \Fee{Some really lengthy and utterly
       		tedious negotiation}     {1818.00}   {12}

   \ProjectTitle{Sales Restructuring}%
       %    Contents                    Rate/Unit   Count
       \Fee{Sales Structure Analysis}   {2750.00}   { 6}
       \Fee{Negotiation with Agents}    {2750.00}   { 3}
\end{invoice}
\end{verbatim}

And here is how the result looks like:

\begin{invoice}{DM}{16}
	\ProjectTitle{Project Phenix}%
	\Fee{Some really lengthy and utterly 
		tedious negotiation}	{1818.00}	{12}

	\ProjectTitle{Sales Restructuring}%
        \Fee{Sales Structure Analysis}   {2750.00}   { 6}
        \Fee{Negotiation with Agents}    {2750.00}   { 3}
\end{invoice}

Hints: If the base currency is to contain a dollar sign (\$), then
dollar sign must be entered in the form of \verb|\string$|
or otherwise the command writing the log file data will fail. The
contents of each fee may be verbose; while the column width is
limited, text contents longer than the column width wraps over
several columns, if necessary.

The astute observer will note that a line beginning with ``Subtotal~Fees''
appeared in the output of the Sales Restructuring Project without explicit
input to this effect from the user's side. The full grammar of the Fee block
requires that all fees are closed by a fee subtotal. Internally,
\texttt{invoice} is defined as a finite state automaton providing
mechanisms to insert a fee subtotal if logic requires it, and print
its value if it makes sense to humans, which is the case if there is more
than one fee.

Note that there is an explicit command \texttt{\char92 STFee} which will
produce a subtotal of the fees charged so far. This can be used if
you want to show fee subtotals within the same project.


\subsection{Invoices without VAT}

As mentioned above, an invoice may be calculated without any VAT.
The VAT lines may completely disappear, as in the following
example:
\begin{verbatim}
\begin{invoice}{DM}{0}
   \ProjectTitle{Project Phenix}%
       %    Contents                    Rate/Unit   Count
       \Fee{Some really lengthy and utterly
       		tedious negotiation}     {1818.00}   {12}

   \ProjectTitle{Sales Restructuring}%
       %    Contents                    Rate/Unit   Count
       \Fee{Sales Structure Analysis}   {2750.00}   { 6}
       \Fee{Negotiation with Agents}    {2750.00}   { 3}
\end{invoice}
\end{verbatim}


Figure~\ref{invoice_without_VAT} on
page~\pageref{invoice_without_VAT}
shows the result.

\begin{figure}[h]
\begin{invoice}{DM}{0}
	\ProjectTitle{Project Phenix}%
	\Fee{Some really lengthy and utterly 
		tedious negotiation}	{1818.00}	{12}

	\ProjectTitle{Sales Restructuring}%
        \Fee{Sales Structure Analysis}   {2750.00}   { 6}
        \Fee{Negotiation with Agents}    {2750.00}   { 3}
\end{invoice}
\caption{A complete invoice
		with fees but no VAT.\label{invoice_without_VAT}}
\end{figure}

In contrast, if for any reason the VAT subtotals and totals should
be kept visible despite a \textit{value} of zero, then use a VAT
value of \verb-0.0- as in the following example:
\begin{verbatim}
\begin{invoice}{DM}{0.0}
   \ProjectTitle{Project Phenix}%
       %    Contents                    Rate/Unit   Count
       \Fee{Some really lengthy and utterly
       		tedious negotiation}     {1818.00}   {12}

   \ProjectTitle{Sales Restructuring}%
       %    Contents                    Rate/Unit   Count
       \Fee{Sales Structure Analysis}   {2750.00}   { 6}
       \Fee{Negotiation with Agents}    {2750.00}   { 3}
\end{invoice}
\end{verbatim}


Figure~\ref{invoice_with_zero_VAT} on
page~\pageref{invoice_with_zero_VAT}
shows the result.

\begin{figure}[h]
\begin{invoice}{DM}{0.0}
	\ProjectTitle{Project Phenix}%
	\Fee{Some really lengthy and utterly 
		tedious negotiation}	{1818.00}	{12}

	\ProjectTitle{Sales Restructuring}%
        \Fee{Sales Structure Analysis}   {2750.00}   { 6}
        \Fee{Negotiation with Agents}    {2750.00}   { 3}
\end{invoice}
\caption{A complete invoice
		with fees but zero VAT.\label{invoice_with_zero_VAT}}
\end{figure}


\section{The Second Example: How to Claim Expenses}

Expenses can be charged in base currency or in any foreign currency.
The base currency's name should be announced once at the beginning
of the invoice.

\subsection{Expenses in Base Currency}

The shape of an expense item in base currency is simple:

\begin{verbatim}
\EBC{<Contents>}                {<Amount>}
\end{verbatim}

Both fields contain mandatory arguments:

\begin{enumerate}
	\item \texttt{<Contents>} contains a description of the
		charged item, e.\,g. ``Hotel'', ``Airport Tax'' or
		whatever.
	\item \texttt{<Amount>} contains the amount in base currency
		units.
\end{enumerate}


\subsection{Expenses in Foreign Currency}

Charging an expense in foreign currency is only slightly more
complicated. The command is:

\begin{verbatim}
\EFC{<Contents>}
    {<Foreign Currency>}{<Amount>}
    {<Conversion Rate>}{<Base Currency Result>}
\end{verbatim}

Arguments to the five fields are partially mandatory, partially optional:
\begin{enumerate}
	\item \texttt{<Contents>} contains a description of the
		charged item, e.\,g. ``Hotel'', ``Airport Tax'' or
		whatever.

	\item \texttt{<Foreign Currency>} contains the name of the
		foreign currency.

	\item \texttt{<Amount>} contains the amount in foreign currency
		units.
	
	\item \texttt{<Conversion Rate>} contains the factor by
		which the foreign currency amount has to be
		multiplied in order to achieve the base currency
		result. If the base currency result is stated,
		then, and only then, the Conversion Rate can be
		omitted.
	
	\item \texttt{<Base Currency Result>} contains an optional
		amount in base currency units. Credit card billing
		statements show this amount which usually contains
		certain service charges of the credit card issuer;
		the base currency result is thus the true amount of
		money to be charged. If a \texttt{<Conversion Rate>}
		is given, stating a base currency result becomes
		optional. This is usually applied for expenses made
		with cash money.
\end{enumerate}


Since some of the arguments given to \verb-\EFC- are optional,
there are basically two different forms of using this command.
With the variant

\begin{verbatim}
\EFC{<Contents>}{<Foreign Currency>}{<Amount>}
        {<Conversion Rate>}               % Conversion rate
        {}                                % Base currency empty!
\end{verbatim}

(amount in foreign currency given, as well as exchange rate stated),
the command will automatically calculate the resulting amount in
base currency.

\begin{verbatim}
\EFC{<Contents>}
        {<Foreign Currency>}{<Amount>}
        {}                                % Conversion rate empty!
        {<Base Currency Result>}          % Base currency
\end{verbatim}

If, however, the exchange rate is omitted and the target amount in
base currency is given, then this value is taken directly. Stating
the resulting amount overrides the internal calculation mechanism.
Examples are given below. We use our interim manager's invoice
again, assuming this time that she spent working on Project Phenix 
12 days in her home country while the Sales Restructuring effort
took her to Hong Kong. The taxi bills are paid in cash, hence she
enters the (fictive) conversion rate, whereas the hotel is paid by
credit card. She can then take the final amount from her credit card
billing statement; an example input would look as follows:

\begin{verbatim}
\begin{invoice}{DM}{16}
   \ProjectTitle{Project Phenix}%
       %    Contents                    Rate/Unit   Count
       \Fee{Negotiation}                {1818.00}   {12}
       %
       %    Contents                    Amount
       \EBC{Hotel, 12 nights}           {2400.00}

   \ProjectTitle{Sales Restructuring}%
       %    Contents                    Rate/Unit   Count
       \Fee{Sales Structure Analysis}   {2750.00}   { 6}
       \Fee{Negotiation with Agents}    {2750.00}   { 3}
       %
       %    Contents              Currency  Amount  Conv.Rate Result
       \EFC{Taxi Airport -- Hotel} {HK\$}  {325.00} {0.2354}   {}
       \EFC{Hotel, 9 nights}       {HK\$}  {9180.00}   {}    {2111.40}
\end{invoice}
\end{verbatim}

Figure~\ref{full_invoice} on
page~\pageref{full_invoice}
shows the result.

\begin{figure}[h]
\begin{invoice}{DM}{16}
	\ProjectTitle{Project Phenix}%
	\Fee{Negotiation}		{1818.00}	{12}
       %
       %    Contents                    Amount
       \EBC{Hotel, 12 nights}		{2400.00}

	\ProjectTitle{Sales Restructuring}%
       \Fee{Sales Structure Analysis}   {2750.00}   { 6}
       \Fee{Negotiation with Agents}    {2750.00}   { 3}
       %
       %    Contents              Currency  Amount  Conv.Rate Result
       \EFC{Taxi Airport -- Hotel} {HK\$}  {325.00} {0.2354}   {}
       \EFC{Hotel, 9 nights}       {HK\$}  {9180.00}   {} {2111.40}
\end{invoice}
\caption{A complete invoice with fees and expenses.\label{full_invoice}}
\end{figure}

Again, a subtotal of the expenses appears only if there is more than
one expense item in a project.


\subsection{Hidden Expense Details}

Occasionally it may be desirable to list only the total of expenses
incurred, while hiding the structure of expenses. In such a case,
the expense commands can be modified by attaching an \verb|i| to
their names (as in \textit{invisible}).

A hidden expense in base currency is thus declared as:

\begin{verbatim}
\EBCi{<Contents>}                {<Amount>}
\end{verbatim}

A hidden expense in foreign currency is declared as:

\begin{verbatim}
\EFCi{<Contents>}
     {<Foreign Currency>}{<Amount>}
     {<Conversion Rate>}{<Base Currency Result>}
\end{verbatim}

Both commands accept and require exactly the same arguments as their
visible counterparts. The expenses will be added to the expense subtotals
and the invoice total in the same way as if they were visible. If there
is no visible expense stated at all it may be useful to have at least 
a line with the expense subtotal appear; this is done by the \verb|STExpenses|
command, as shown in the following example:

\begin{verbatim}
\begin{invoice}{DM}{16}
   \ProjectTitle{Project Phenix}%
       %    Contents                    Rate/Unit   Count
       \Fee{Negotiation}                {1818.00}   {12}
       %
       %    Contents                    Amount
       \EBCi{Hotel, 12 nights}           {2400.00}
       %
       \STExpenses

   \ProjectTitle{Sales Restructuring}%
       %    Contents                    Rate/Unit   Count
       \Fee{Sales Structure Analysis}   {2750.00}   { 6}
       \Fee{Negotiation with Agents}    {2750.00}   { 3}
       %
       %    Contents              Currency  Amount  Conv.Rate Result
       \EFCi{Taxi Airport -- Hotel} {HK\$}  {325.00} {0.2354}   {}
       \EFCi{Hotel, 9 nights}       {HK\$}  {9180.00}   {}    {2111.40}
       \STExpenses
\end{invoice}
\end{verbatim}

Figure~\ref{invoice_with_hidden_expenses} on
page~\pageref{invoice_with_hidden_expenses}
shows the result.

\begin{figure}[h]
\begin{invoice}{DM}{16}
       \ProjectTitle{Project Phenix}%
       \Fee{Negotiation}		{1818.00}	{12}
       %
       %    Contents                    Amount
       \EBCi{Hotel, 12 nights}		{2400.00}
       % 
       \STExpenses

       \ProjectTitle{Sales Restructuring}%
       \Fee{Sales Structure Analysis}   {2750.00}   { 6}
       \Fee{Negotiation with Agents}    {2750.00}   { 3}
       %
       %    Contents              Currency  Amount  Conv.Rate Result
       \EFCi{Taxi Airport -- Hotel} {HK\$}  {325.00} {0.2354}   {}
       \EFCi{Hotel, 9 nights}       {HK\$}  {9180.00}   {} {2111.40}
       \STExpenses
\end{invoice}
\caption{A complete invoice with visible fees
		but invisible expenses.\label{invoice_with_hidden_expenses}}
\end{figure}


\section{Deduction of Discounts and Downpayments}

Finally, it may be possible that previous downpayments etc. have to 
be considered for the invoice total. An optional discount line
can state the item and deduct the amount from the invoice total.
Deductions are always stated in base currency. They follow the following
syntax:


\begin{verbatim}
\Discount{<Contents>}     {<Amount>}
\end{verbatim}

Both fields contain mandatory arguments:

\begin{enumerate}
	\item \texttt{<Contents>} contains a description of the
		discount item, e.\,g. ``Downpayment'', ``Tickets for 
		private use'' or whatever.
	\item \texttt{<Amount>} contains the amount in base currency
		units.
\end{enumerate}

Figure~\ref{full_and_discounted_invoice} on
page~\pageref{full_and_discounted_invoice}
shows the result.

\begin{figure}[h]
\begin{invoice}{DM}{16}
	\ProjectTitle{Project Phenix}%
	\Fee{Negotiation}		{1818.00}	{12}
       %
       %    Contents                    Amount
       \EBC{Hotel, 12 nights}		{2400.00}

	\ProjectTitle{Sales Restructuring}%
       \Fee{Sales Structure Analysis}   {2750.00}   { 6}
       \Fee{Negotiation with Agents}    {2750.00}   { 3}
       %
       %    Contents              Currency  Amount  Conv.Rate Result
       \EFC{Taxi Airport -- Hotel} {HK\$}  {325.00} {0.2354}   {}
       \EFC{Hotel, 9 nights}       {HK\$}  {9180.00}   {} {2111.40}
       \Discount{Downpayment received}  {2000.00}
\end{invoice}
\caption{A complete invoice with fees,
		expenses and a deduction.\label{full_and_discounted_invoice}}
\end{figure}

\section{Postprocessing}

In order to allow the further processing of
\texttt{invoice}-generated data, the log file contains the totals of
fees, expenses and taxes in the form of \verb-key:value- pairs. This
information appears also on the terminal while \LaTeXe\ runs. Please
note that the key appearing in the log file is expressed in the same
language as that of the master document.

\section{Document Language}

All column headers appearing in \texttt{invoice} can be redefined in
order to match the language of the master document. E.\,g., headings
like ``Total'' will automatically appear as ``Gesamtsumme'' if the
document language is set to German (either via \texttt{babel} or
\verb-\usepackage{german}-).

Please observe that the \texttt{invoice} package must be called {\em
after} the document language has been selected.

This is correct:

\begin{verbatim}
\documentclass[10pt]{ltxdoc}
\usepackage{german}
\usepackage{invoice} % labels will now appear in German!
\end{verbatim}

This will not work:

\begin{verbatim}
\documentclass[10pt]{ltxdoc}
\usepackage{invoice}
\usepackage{german} % labels will still appear in English!
\end{verbatim}


\subsection{Specifying the Desired Language}

It is also possible to ignore the global language settings and select
a language local to the \texttt{invoice} package by specifying the
desired language as an option:

\begin{verbatim}
\documentclass[10pt]{ltxdoc}
\usepackage[german]{invoice} % labels in German, rest 
                             % of document in English
\end{verbatim}

This is useful whenever the desired language is not provided by the
installation base where \texttt{invoice} is executed.

\subsection{Adding Labels in a New Language}

At present, column labels produced by \texttt{invoice} can appear
in six languages: Afrikaans, Dutch, English, French, German and Italian.
Other languages can be added easily by editing the file
\verb-invoice.def-. Language-dependent definitions are contained in
the \verb-\if-clause. Extending the file is simple: 

\begin{enumerate}
	\item Copy the labels template to the end of the file.
	\item Remove the comments (\verb-%-) in column 1.
	\item Fill each label definition with the appropriate
		foreign language translation.
	\item Put the correct internal name of the foreign language
		into the condition expression of the \verb-\ifx--
		and \verb-\ifnum--clauses.
\end{enumerate}

\begin{sloppypar}
Whenever you create your own foreign language extension please kindly
consider sending your modified \verb-invoice.def- file to the author
(at \verb+corff@zedat.fu-berlin.de+ or \verb+oliver.corff@email.de+)
so that it can be included in the next update. Future users can then
share your work.
\end{sloppypar}

\section{Bugs}

Certainly there are bugs. After all, this is not Moon, but Earth
(where life is supposed to exist, as opposed to Moon). The author 
has not conducted extensive tests on the algebraic functions, and
there may be rounding and truncating errors.

The author considers it a bug that the decimal point cannot be
replaced by a comma at present, while entering dollar signs
requires the \verb|\string$| notation.

Spurious spaces may distort the table layout. It is especially
recommended to close all \verb"\ProjectTitle{...}%" lines with a
percent mark, as shown. If the percent mark is omitted, the first
column header is not properly aligned to the left.

Breaking long invoice statements with many projects over several
pages results in many orphans and widows.

\subsection{\label{koma}Compatibility with KOMA-Script}

Unfortunately, when first writing this package the author was not aware
of the KOMA-Script classes, notably the class \texttt{scrlettr.cls}
which defines its own \verb-\invoice{#1}- command, albeit with completely
different scope, usage and syntax. KOMA-Script's \verb-\invoice{#1}- command
accepts a number and prints ``Invoice no. \#1'' in a letter 
opening. Now that the names of the \texttt{invoice} package and
environment have stuck it is easier to think of a workaround than to
conceive a new name. Therefore, If the class \texttt{scrlettr.cls} is
used, the environment \texttt{invoice} is automatically renamed to
\texttt{invoiceenv}. Writing 

\begin{verbatim}
\documentclass[10pt]{scrlettr}
\usepackage{invoice}
\begin{invoice}{DM}{16}
...
...
\end{verbatim}

will result in an error message. Say \verb-\begin{invoiceenv}{DM}{16}-
instead.  KOMA-Script's own \verb-\invoice{#1}- command is renamed to
\verb-\invoiceno{#1}-, while \verb-\invoice{#1}- now generates an 
error message reminding KOMA-Script users to apply the renamed
commands.

\section{Desiderata}

The \verb-invoice- package is far from complete. Future expansions
should aim at making the package more flexible. How taxes are
treated, etc., is at present a rather particular arrangement
suitable for the author's immediate needs, but a more general
solution can be designed as soon as more users reflect their needs
to the author.
\end{document}
